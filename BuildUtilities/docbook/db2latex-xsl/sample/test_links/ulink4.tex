\documentclass[pdftex,a4paper,10pt]{article}
\usepackage[pdftex,bookmarksnumbered,colorlinks,backref, bookmarks, breaklinks, linktocpage,pdfstartview=FitH,pdfpagemode=UseNone]{hyperref}
\usepackage[pdftex]{graphicx}
\usepackage[latin1]{inputenc}
\usepackage{times}
\usepackage{varioref}
\setlength{\textwidth}{11cm}
\pdfcompresslevel=9
% In any other LaTeX context, this would probably go into a style file.
\newcommand{\docbookhyphenateurl}[1]{{\hyphenchar\font=`\/\relax #1\hyphenchar\font=`\-}}
\newcommand{\docbookhyphenatedot}[1]{{\hyphenchar\font=`.\relax #1\hyphenchar\font=`\-}}
% --------------------------------------------
\makeatletter
\newcommand{\dbz}{\penalty \z@}
\newcommand{\docbooktolatexpipe}{\ensuremath{|}\dbz}
\usepackage[latin1]{inputenc}

\def\docbooktolatexgobble{\expandafter\@gobble}
% Facilitate use of \cite with \label
\newcommand{\docbooktolatexbibaux}[2]{%
  \protected@write\@auxout{}{\string\global\string\@namedef{docbooktolatexcite@#1}{#2}}
}
\newcommand{\docbooktolatexcite}[2]{%
  \@ifundefined{docbooktolatexcite@#1}%
  {\cite{#1}}%
  {\def\@docbooktolatextemp{#2}\ifx\@docbooktolatextemp\@empty%
   \cite{\@nameuse{docbooktolatexcite@#1}}%
   \else\cite[#2]{\@nameuse{docbooktolatexcite@#1}}%
   \fi%
  }%
}
\newcommand{\docbooktolatexbackcite}[1]{%
  \ifx\Hy@backout\@undefined\else%
    \@ifundefined{docbooktolatexcite@#1}{%
      % emit warning?
    }{%
      \ifBR@verbose%
        \PackageInfo{backref}{back cite \string`#1\string' as \string`\@nameuse{docbooktolatexcite@#1}\string'}%
      \fi%
      \Hy@backout{\@nameuse{docbooktolatexcite@#1}}%
    }%
  \fi%
}
% Provide support for bibliography `subsection' environments with titles
\newenvironment{docbooktolatexbibliography}[3]{
   \begingroup
   \let\save@@chapter\chapter
   \let\save@@@mkboth\@mkboth
   \let\save@@bibname\bibname
   \let\@mkboth\@gobbletwo
   \def\@tempa{#3}
   \def\@tempb{}
   \ifx\@tempa\@tempb
      \let\chapter\@gobbletwo
      \let\bibname\relax
   \else
      \def\chapter{\subsection}
      \def\bibname{\@tempa}
   \fi
   \begin{thebibliography}{#1}
}{
   \end{thebibliography}
   \let\chapter\save@@chapter
   \let\@mkboth\save@@@mkboth
   \let\bibname\save@@bibname
   \endgroup
}
% Prevent multiple openings of the same aux file
% (happens when backref is used with multiple bibliography environments)
\ifx\AfterBeginDocument\undefined\let\AfterBeginDocument\AtBeginDocument\fi
\AfterBeginDocument{
   \let\latex@@starttoc\@starttoc
   \def\@starttoc#1{%
      \@ifundefined{docbooktolatex@aux#1}{%
         \global\@namedef{docbooktolatex@aux#1}{}%
         \latex@@starttoc{#1}%
      }{}
   }
}
% --------------------------------------------
% A way to honour <footnoteref>s
% Blame j-devenish (at) users.sourceforge.net
% In any other LaTeX context, this would probably go into a style file.
\newcommand{\docbooktolatexusefootnoteref}[1]{\@ifundefined{@fn@label@#1}%
  {\hbox{\@textsuperscript{\normalfont ?}}%
    \@latex@warning{Footnote label `#1' was not defined}}%
  {\@nameuse{@fn@label@#1}}}
\newcommand{\docbooktolatexmakefootnoteref}[1]{%
  \protected@write\@auxout{}%
    {\global\string\@namedef{@fn@label@#1}{\@makefnmark}}%
  \@namedef{@fn@label@#1}{\hbox{\@textsuperscript{\normalfont ?}}}%
  }
% --------------------------------------------
% Hacks for honouring row/entry/@align
% (\hspace not effective when in paragraph mode)
% Naming convention for these macros is:
% 'docbooktolatex' 'align' {alignment-type} {position-within-entry}
% where r = right, l = left, c = centre
\newcommand{\docbooktolatex@align}[2]{\protect\ifvmode#1\else\ifx\LT@@tabarray\@undefined#2\else#1\fi\fi}
\newcommand{\docbooktolatexalignll}{\docbooktolatex@align{\raggedright}{}}
\newcommand{\docbooktolatexalignlr}{\docbooktolatex@align{}{\hspace*\fill}}
\newcommand{\docbooktolatexaligncl}{\docbooktolatex@align{\centering}{\hfill}}
\newcommand{\docbooktolatexaligncr}{\docbooktolatex@align{}{\hspace*\fill}}
\newcommand{\docbooktolatexalignrl}{\protect\ifvmode\raggedleft\else\hfill\fi}
\newcommand{\docbooktolatexalignrr}{}
\ifx\captionswapskip\@undefined\newcommand{\captionswapskip}{}\fi
% index labelling helper
\def\docbooktolatexprintindex#1{%
   \let\docbooktolatexindexname\indexname%
   \def\indexname{\label{#1}\let\indexname\docbooktolatexindexname\indexname}%
   \printindex}
\makeatother
\title{\textbf{Links}}
\author{Ramon Casellas \and James Devenish}
\begin{document}
{\maketitle\pagestyle{empty}\thispagestyle{empty}}

% ------------------------   
% Section 
\section{Plain Tests}
\label{id223678}\hypertarget{id223678}{}%

buf\_init: \hyperlink{buf-init}{buf\_init},
Third Section:
\hyperlink{sec:ulinks}{Third Section},
Ulinks\_:
\hyperlink{sec:ulinks}{Ulinks\_},
Section 3:
\hyperlink{sec:ulinks}{Section {\ref{sec:ulinks}}}.
Title:
\hyperlink{title:ulinks}{Title}.
% ------------------------   
% Section 
\section{Cross-reference Tests}
\label{id184727}\hypertarget{id184727}{}%

Ulinks\_:
\hyperlink{sec:ulinks}{Ulinks\_},
Section 3:
\hyperlink{sec:ulinks}{Section {\ref{sec:ulinks}}},
XrefId[???]:
XrefId[??].
% ------------------------   
% Section 
\section{Ulinks\_}
\label{sec:ulinks}\hypertarget{sec:ulinks}{}%
\hypertarget{buf-init}{}
This is an amount of text that may make this sentence long \url{http://www.enst.fr/#top}

This is might also make this sentence a bit long \url{http://www.enst.fr/\string&_~#top}
\subsection{{\texttt{{More/\dbz{}more/\dbz{} more/\dbz{}more/\dbz{} more/\dbz{}more/\dbz{} aaaa/\dbz{}aaaa/\dbz{}b/\dbz{}b/\dbz{}b/\dbz{}b/\dbz{}cccc/\dbz{}cccc/\dbz{} more/\dbz{}more/\dbz{} more/\dbz{}more}}}}
\label{id185427}\hypertarget{id185427}{}%

http://www.enst.fr\label{id185441}\begingroup\catcode`\#=12\footnote{ \url{http://www.enst.fr/\string&x}}\endgroup\docbooktolatexmakefootnoteref{id185441} \url{http://www.enst.fr/\string&x}

This is might also make this sentence long enough for the regular text blah http://www.enst.fr/\_\textasciitilde{}\#top {\texttt{{http:/\dbz{}/\dbz{}www.\dbz{}enst.\dbz{}fr/\dbz{}\_\dbz{}\textasciitilde{}\#top}}}\label{id185449}\begingroup\catcode`\#=12\footnote{ \url{http://www.enst.fr/#top}}\endgroup\docbooktolatexmakefootnoteref{id185449} \url{http://www.enst.fr/#top}

ENST\label{id185462}\begingroup\catcode`\#=12\footnote{ \url{http://www.enst.fr}}\endgroup\docbooktolatexmakefootnoteref{id185462} \url{http://www.enst.fr}

\texttt{<\url{email}\label{id185468}\begingroup\catcode`\#=12\footnote{ \url{mailto:email}}\endgroup\docbooktolatexmakefootnoteref{id185468}>}

Regular text \_\textasciitilde{}\#1 with footnote\label{id185474}\begingroup\catcode`\#=12\footnote{
\url{http://www.enst.fr/#top}

\url{http://www.enst.fr/\string&_~#top}

http://www.enst.fr \url{http://www.enst.fr/\string&x}

blah http://www.enst.fr/\_\textasciitilde{}\#top {\texttt{{http:/\dbz{}/\dbz{}www.\dbz{}enst.\dbz{}fr/\dbz{}\_\dbz{}\textasciitilde{}\#top}}} \url{http://www.enst.fr/#top}

ENST \url{http://www.enst.fr/_~#top}

\texttt{<\url{email}>}

Regular text \_\textasciitilde{}\#2.
}\endgroup\docbooktolatexmakefootnoteref{id185474}.

% --------------------------------------------
% End of document
% --------------------------------------------
\end{document}
