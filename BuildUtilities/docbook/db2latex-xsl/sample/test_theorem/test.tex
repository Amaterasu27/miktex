% ------------------------------------------------------------	
% Autogenerated LaTeX file for books	
% ------------------------------------------------------------	
\ifx\pdfoutput\undefined
\documentclass[french,english,a4paper,12pt,]{report}
\else
\documentclass[pdftex,french,english,a4paper,12pt,]{report}
\fi
\label{book}\usepackage{ifthen}
% --------------------------------------------
% Check for PDFLaTeX/LaTeX 
% --------------------------------------------
\newif\ifpdf
\ifx\pdfoutput\undefined
\pdffalse % we are not running PDFLaTeX
\else
\pdfoutput=1 % we are running PDFLaTeX
\pdftrue
\fi
% --------------------------------------------
% Load graphicx package with pdf if needed 
% --------------------------------------------
\ifpdf
\usepackage[pdftex]{graphicx}
\pdfcompresslevel=9
\else
\usepackage{graphicx}
\fi
\usepackage{anysize}
\marginsize{3cm}{2cm}{1.25cm}{1.25cm}
\usepackage{fancyhdr}
\pagestyle{fancy}\renewcommand{\headrulewidth}{0.4pt}
\renewcommand{\footrulewidth}{0.4pt}
% ---------------------- 
% Most Common Packages   
% ---------------------- 
\usepackage{makeidx} 
\usepackage{varioref}         
\usepackage{latexsym}         
\usepackage{enumerate}         
\usepackage{fancybox}      
\usepackage{float}       
\usepackage{ragged2e}       
\usepackage[french]{babel} 
\usepackage{fancyvrb}         
\makeatletter\@namedef{FV@fontfamily@default}{\def\FV@FontScanPrep{}\def\FV@FontFamily{}}\makeatother
\fvset{obeytabs=true,tabsize=3}
\usepackage{isolatin1}         
\usepackage{rotating}         
\usepackage{subfigure}         
\usepackage{tabularx}         
\usepackage{url}         
% ---------------
% Document Font  
% ---------------
\usepackage{palatino}
 \def\keywords{\vspace{-.3em}
 \if@twocolumn
 \small{\itshape 
Keywords }\/\bfseries---$\!$%
 \else
 \begin{center}\small\bfseries 
Keywords \end{center}\quotation\small
 \fi}
 \def\endkeywords{\vspace{0.6em}\par\if@twocolumn\else\endquotation\fi
 \normalsize\rmfamily}
% --------------------------------------------
% Math support                                
% --------------------------------------------
\usepackage{amsmath,amsthm, amsfonts, amssymb, amsxtra,amsopn}
%\newtheorem{thm}{Theorem}[section]
%\newtheorem{cor}[section]{Corollary}
%\newtheorem{lem}[section]{Lemma}
%\newtheorem{defn}[section]{Definition}
%\newtheorem{prop}[section]{Proposition}
%\newtheorem{ax}{Axiom}
%\newtheorem{theorem}[section]{Theorem}
%\newtheorem{corollary}{Corollary}
%\newtheorem{lemma}{Lemma}
%\newtheorem{proposition}{Proposition}
%\theoremstyle{definition}
%\newtheorem{definition}{Definition}
%\theoremstyle{remark}
%\newtheorem{rem}{Remark}
%\newtheorem*{notation}{Notation}
%\newcommand{\ntt}{\normalfont\ttfamily}
%\newcommand{\thmref}[1]{Theorem~\ref{#1}}
%\newcommand{\secref}[1]{\S\ref{#1}}
%\newcommand{\lemref}[1]{Lemma~\ref{#1}}
 \newcommand{\bysame}{\mbox{\rule{3em}{.4pt}}\,}
 \newcommand{\A}{\mathcal{A}}
 \newcommand{\B}{\mathcal{B}}
 \newcommand{\XcY}{{(X,Y)}}
 \newcommand{\SX}{{S_X}}
 \newcommand{\SY}{{S_Y}}
 \newcommand{\SXY}{{S_{X,Y}}}
 \newcommand{\SXgYy}{{S_{X|Y}(y)}}
 \newcommand{\Cw}[1]{{\hat C_#1(X|Y)}}
 \newcommand{\G}{{G(X|Y)}}
 \newcommand{\PY}{{P_{\mathcal{Y}}}}
 \newcommand{\X}{\mathcal{X}}
 \newcommand{\wt}{\widetilde}
 \newcommand{\wh}{\widehat}
 % --------------------------------------------
 %\DeclareMathOperator{\per}{per}
 \DeclareMathOperator{\cov}{cov}
 \DeclareMathOperator{\non}{non}
 \DeclareMathOperator{\cf}{cf}
 \DeclareMathOperator{\add}{add}
 \DeclareMathOperator{\Cham}{Cham}
 \DeclareMathOperator{\IM}{Im}
 \DeclareMathOperator{\esssup}{ess\,sup}
 \DeclareMathOperator{\meas}{meas}
 \DeclareMathOperator{\seg}{seg}
% --------------------------------------------
% --------------------------------------------
% Load hyperref package with pdf if needed 
% --------------------------------------------
\ifpdf
\usepackage[pdftex,bookmarksnumbered,colorlinks,backref, bookmarks, breaklinks, linktocpage,pdfstartview=FitH]{hyperref}
\else
\usepackage[dvips,bookmarksnumbered,colorlinks,backref, bookmarks, breaklinks, linktocpage,]{hyperref}
\fi
% --------------------------------------------
\newenvironment{admminipage}{\begin{Sbox}\begin{minipage}}{\end{minipage}\end{Sbox}\fbox{\TheSbox}}
\newlength{\admlength}
\newenvironment{admonition}[2] {
 \hspace{0mm}\newline\hspace*\fill\newline
 \noindent
 \setlength{\fboxsep}{5pt}
 \setlength{\admlength}{\linewidth}
 \addtolength{\admlength}{-10\fboxsep}
 \addtolength{\admlength}{-10\fboxrule}
 \admminipage{\admlength}
 {\bfseries \sc\large{#2}} \newline
 \\[1mm]
 \sffamily
 \includegraphics[width=1cm]{#1}
 \addtolength{\admlength}{-1cm}
 \addtolength{\admlength}{-20pt}
 \begin{minipage}[lt]{\admlength}
 \parskip=0.5\baselineskip \advance\parskip by 0pt plus 2pt
}{
 \vspace{5mm} 
 \end{minipage}
 \endadmminipage
 \vspace{.5em}
 \par
}
% --------------------------------------------
% Commands to manage/style/create floats      
% figures, tables, algorithms, examples, eqn  
% --------------------------------------------
 \floatstyle{ruled}
 \restylefloat{figure}
 \floatstyle{ruled}
 \restylefloat{table}
 \floatstyle{ruled}
 \newfloat{program}{ht}{lop}[section]
 \floatstyle{ruled}
 \newfloat{example}{ht}{loe}[section]
 \floatname{example}{Example}
 \floatstyle{ruled}
 \newfloat{dbequation}{ht}{loe}[section]
 \floatname{dbequation}{Equation}
 \floatstyle{boxed}
 \newfloat{algorithm}{ht}{loa}[section]
 \floatname{algorithm}{Algorithm}
\ifpdf
\DeclareGraphicsExtensions{.pdf,.png,.jpg}
\else
\DeclareGraphicsExtensions{.eps}
\fi
% --------------------------------------------
% $latex.caption.swapskip enabled for $formal.title.placement support
\newlength{\docbooktolatextempskip}
\newcommand{\captionswapskip}{\setlength{\docbooktolatextempskip}{\abovecaptionskip}\setlength{\abovecaptionskip}{\belowcaptionskip}\setlength{\belowcaptionskip}{\docbooktolatextempskip}}
% --------------------------------------------
\makeatletter
\def\docbooktolatexgobble{\expandafter\@gobble}
% Facilitate use of \cite with \label
\newcommand{\docbooktolatexbibaux}[2]{%
  \protected@write\@auxout{}{\string\global\string\@namedef{docbooktolatexcite@#1}{#2}}
}
\newcommand{\docbooktolatexcite}[2]{%
  \@ifundefined{docbooktolatexcite@#1}%
  {\cite{#1}}%
  {\def\@docbooktolatextemp{#2}\ifx\@docbooktolatextemp\@empty%
   \cite{\@nameuse{docbooktolatexcite@#1}}%
   \else\cite[#2]{\@nameuse{docbooktolatexcite@#1}}%
   \fi%
  }%
}
\newcommand{\docbooktolatexbackcite}[1]{%
  \ifx\Hy@backout\@undefined\else%
    \@ifundefined{docbooktolatexcite@#1}{%
      % emit warning?
    }{%
      \ifBR@verbose%
        \PackageInfo{backref}{back cite \string`#1\string' as \string`\@nameuse{docbooktolatexcite@#1}\string'}%
      \fi%
      \Hy@backout{\@nameuse{docbooktolatexcite@#1}}%
    }%
  \fi%
}
% --------------------------------------------
% A way to honour <footnoteref>s
% Blame j-devenish (at) users.sourceforge.net
% In any other LaTeX context, this would probably go into a style file.
\newcommand{\docbooktolatexusefootnoteref}[1]{\@ifundefined{@fn@label@#1}%
  {\hbox{\@textsuperscript{\normalfont ?}}%
    \@latex@warning{Footnote label `#1' was not defined}}%
  {\@nameuse{@fn@label@#1}}}
\newcommand{\docbooktolatexmakefootnoteref}[1]{%
  \protected@write\@auxout{}%
    {\global\string\@namedef{@fn@label@#1}{\@makefnmark}}%
  \@namedef{@fn@label@#1}{\hbox{\@textsuperscript{\normalfont ?}}}%
  }
% --------------------------------------------
% Hacks for honouring row/entry/@align
% (\hspace not effective when in paragraph mode)
% Naming convention for these macros is:
% 'docbooktolatex' 'align' {alignment-type} {position-within-entry}
% where r = right, l = left, c = centre
\newcommand{\docbooktolatexalignrl}{\protect\ifvmode\raggedleft\else\hfill\fi}
\newcommand{\docbooktolatexalignrr}{\protect}
\newcommand{\docbooktolatexalignll}{\protect\ifvmode\raggedright\else\fi}
\newcommand{\docbooktolatexalignlr}{\protect\ifvmode\else\hspace*\fill\fi}
\newcommand{\docbooktolatexaligncl}{\protect\ifvmode\centering\else\hfill\fi}
\newcommand{\docbooktolatexaligncr}{\protect\ifvmode\else\hspace*\fill\fi}
\ifx\captionswapskip\@undefined\newcommand{\captionswapskip}{}\fi
\makeatother

\bibliographystyle{ieeetr}
\theoremstyle{plain}
\newtheorem{thm}{Th�r�me}[section]
\theoremstyle{definition}
\newtheorem{defn}{D�finition}[section]
\title{\bfseries Theorems}
\author{John Doe \and Ramon Casellas}
% --------------------------------------------
\makeindex
\makeglossary
% --------------------------------------------

\setcounter{tocdepth}{4}

\setcounter{secnumdepth}{4}
\begin{document}

\InputIfFileExists{title}{\typeout{WARNING: Using cover pagetitle}}{\maketitle\pagestyle{fancy}\thispagestyle{empty}}

% -------------------------------------------------------------
% Chapter This is NOT DOCBOOK 
% ------------------------------------------------------------- 	
\chapter{This is NOT DOCBOOK}
\label{id173363}\hypertarget{id173363}{}%

 I have extended (for my own purposes) the DTD, 
using {\texttt{mathelement}}, and its content model that I saw on a
mailing list (credit is due, contact me).


      \begin{defn}[Task]

        
        
          
 A task is something that has to be done, usually given by 
	your boss, under the hypothesis that you do not want to. (Otherwise
	it is a pleasure, like working on DB2LaTeX).

        
      \end{defn}

    
      \begin{thm}[Lazy man theorem]

        
        
          
Given a task to do, T

        
        
          
Do not perform task T today, if it can be done tomorrow.

        
        \begin{proof}
          
 A proof will be given tomorrow.

        \end{proof}

      \end{thm}

    
      \begin{defn}[Processus stationnaire]

        
        
          
Un processus stochastique $x(t)$ est dit {\bfseries stationnaire} si  $ \forall n \in \mathbb{N}, \forall \tau, \forall t_0 < t_1 < t_2 < \ldots < t_n $ on a :

        
        

\begin{equation*}
( x(t_0), \ldots, x(t_n) ) \quad  =_{\mathbb{L}} \quad (x(t_0 + \tau), \ldots, x(t_n + \tau))
\end{equation*}

        
        

\begin{equation*}
\rho(\tau) = \frac{\textrm{cov}[x(t),x(t+\tau)]} {\sqrt{\textrm{var}[x(t+\tau)] \textrm{var}[x(t)]}}
\end{equation*}

        
        

\begin{itemize}
\item $\mathbb{E}[x(t)] = \lambda < \infty$ 
\item $\mathbb{E}[(x(t) - \lambda)^2] = \sigma^2 < \infty$
\item $\mathbb{E}[(x(t) - \lambda)(x(t+\tau) - \lambda)] = \textrm{cov}(\tau) < \infty$
\end{itemize}

        
        

\begin{equation*}
\rho(\tau) = \frac{\textrm{cov}(\tau)} {\sigma^2}
\end{equation*}

        
      \end{defn}

    
      \begin{defn}[Processus Cumulatif (ang. Cumulant Process)]

        
        
          
Soit $x(t)$ un processus stochastique discret (resp. continu), et $t_0,t_1 \in \mathbb{N}$ (resp. $t_0,t_1 \in \mathbb{R}$). Le processus $X[t_0,t_1) \triangleq \sum_{t_0}^{t_1} x(t)$ (resp. $X[t_0,t_1) \triangleq \int_{t_0}^{t_1}{ x(t) dt}$) est dit processus cumulatif (ou {\bfseries  processus d'accroissements}) de $x(t)$.

        
      \end{defn}

    
      \begin{defn}[Processus � accroissements ind�pendants]

        
        
          
Un processus $x(t)$ est dit {\bfseries � accroissements ind�pendants } si pour n'importe quelle suite d'instants de temps  $0 = t_0 < t_1 < t_2 < \ldots < t_n $, les accroissements du processus $x(t_n) - x(t_{n-1}) , x(t_{n-1}), \ldots, x(t_1) - x(t_0)$ sont ind�pendants.

        
      \end{defn}

    
      \begin{defn}[Processus � borne stationnaire]

        
        
          
 Un processus d'accroissements $x(t)$ est born� stationnairement  si $\forall h$

\begin{equation}
\lim_{a \to \infty} \sup_{t} \mathbb{P} \left\{ x(t+h) - x(t) \ge a \right\} = 0
\end{equation}



        
      \end{defn}

    
      \begin{defn}[Processus �  m�moire longue (ang. Long Range Dependent)]

        
        
          
 Un processus $x(t)$ stationnaire est dit �� m�moire longue� {\em (ang. Long Range Dependent)} si

\begin{equation}
\sum\limits_{k=-\infty}^{\infty} \left| \rho_x(k) \right| = \infty
\end{equation}



        
      \end{defn}

    
      \begin{defn}[Mod�les de Trafic � Queue Lourde]

        
        
          
 Une variable al�atoire $X$ est dit �� queue lourde�  si $\exists \alpha,  0 < \alpha < 2$ et $\exists C$ tel que $x^\alpha \mathbb{P}(|X| > x) \to C$, quand $x \to \infty$, o� $C$ est une constante et $\alpha$ est l'index de la distribution. Un processus avec des distributions marginales � queue lourde est dit un processus � queue lourde.

        
      \end{defn}

    
      \begin{defn}[Auto-similarit�]

        
        
          
Un processus  $x(t)$ est dit �auto similaire�  {\em (self-similar)} de param�tre H, si le processus $c^{-H} x(ct)$ et le processus $x(t)$ sont �quivalents en distribution. L'exemple classique de processus auto similaire est le processus mouvement fractionnaire Brownien (fBm) de param�tre H (param�tre de Hurst). Voir par exemple \docbooktolatexcite{RabEBE}{} (pp. 34) ou \docbooktolatexcite{Norros}{}.

        
      \end{defn}

    % ------------------------------------------- 
%	
%  Bibliography
%	
% -------------------------------------------	
\bibliography{./citations}

\end{document}

