% This is the pdfTeX FAQ.
% It should compile using
% latex FAQNAME
%    or
% pdflatex FAQNAME
%
% It uses an index with a special index style file.
% 1) Make a file faq.ist with the following lines in it:
%      actual '=' % = instead of deafult @
%      item_1 "\t"
%      delim_1 "--"
% 2) makeindex FAQNAME -s faq.ist
%  (sorry had to do it to include the symbol ``@'' in the index)
% rerun: latex FAQNAME
%
% You need url.sty from CTAN//tex-archive/macros/latex/contrib/other/misc
% and while you are at it, you *really* should get all of hyperref.
%
% A PDF and DVI version of this FAQ are available at
% http://www.tug.org/applications/pdftex/
%
\documentclass[11pt]{article}

% are we using pdftex or normaltex
\newif\ifpdf
\ifx\pdfoutput\undefined
   \pdffalse
\else
   \pdfoutput=1
   \pdftrue
\fi

\usepackage{makeidx}
\usepackage{url}
\usepackage{times} % saves a lot of ouptut space in PDF.  Delete if
                   % you cannot get PS fonts working on your system.
\ifpdf
  \usepackage[pdftex,colorlinks=true,urlcolor=blue,pdfstartview=FitH]{hyperref}
  \pdfcompresslevel=9
\fi

\makeindex


\begin{document}

\newcommand{\maintainer}{Jody Klymak \url{mailto:jklymak@apl.washington.edu}}

\newcommand{\question}[3]{\subsubsection{#2}\label{q:#1}}
% \question{label}{question title}{YYYY/MM/DD(last modification)}
\newcommand{\sectionf}[2]{\section{#2} \label{sec:#1}}
\newcommand{\subsectionf}[2]{\subsection{#2} \label{sec:#1}}

\newcommand{\contributor}[3]{\index{#1=#1
    \protect\url{#2}!Q\thesubsubsection}
%    {\scriptsize Contributed by: #1 \url{#2}\\}
}
% \contributor{Name}{email}{YYYY/MM/DD}

\newcommand{\pathf}[1]{\url{#1}}

% make the index environment better....
\renewcommand{\indexname}{Contributors}
\renewenvironment{theindex}{
 \appendix
 \section{\indexname}
 \begin{itemize}
}
{\end{itemize}}

\title{
  The pdfTeX FAQ \\
  Version 0.10}

\author{
  Maintained by: Jody M. Klymak\\
  \url{mailto:jklymak@apl.washington.edu} }

\date{\today}

\maketitle

\begin{abstract}{ This document is a preliminary version of pdfTeX's
    frequently asked questions (FAQ) and answers. If you can
    contribute to this document, please mail Jody Klymak
    (\url{mailto:jklymak@apl.washington.edu}) or pdfTeX's mailing list
    (\url{mailto:pdftex@tug.org}). Including the key \verb+FAQ+ in the
    subject line of your contribution will help the FAQ maintainer
    stay organized.  This document is in TeX syntax. The character set
    is ISO-8859-2.  }
\end{abstract}

\copyright{This document is freely distributable.}

\tableofcontents

\sectionf{General}{ General }

\subsectionf{TeX_and_PDF}{ TeX and PDF }

\question{What_is_pdfTeX}{ What is pdfTeX?  }{1998/09/10}

{
\contributor{Pavel Jan\'{\i}k
    jr}{mailto:Pavel.Janik@inet.cz}{1998/09/10}

  pdfTeX is a variant of well known typesetting program of prof.
  Donald E.  Knuth -- TeX.  Output of Knuth's TeX is a file in DVI
  format. The difference between TeX and pdfTeX is that pdfTeX
  directly generates PDF. You can also create PDF with Adobe's
  Distiller program, using a DVI to PostScript program to create PS
  from TeX's DVI file.

  WARNING: pdfTeX is alpha software!  }


\question{What_is_TeX}{ What is TeX?  }{1998/09/10}

{ \contributor{Pavel Jan\'{\i}k
jr}{mailto:Pavel.Janik@inet.cz}{1998/09/10} From `TeX -- The
Program' by Donald E. Knuth: ``This is TeX, a document compiler
intended to produce typesetting of high quality''. TeX is a batch
oriented typesetting system. When we talk about TeX we mean the
macro programming language as well as the program that interprets
and executed this language.  }


\question{What_is_PDF} { What is PDF?  }{1998/09/10}

{ \contributor{Pavel Jan\'{\i}k jr}
  {mailto:Pavel.Janik@inet.cz}{1998/09/10}

  This question is answered
  in the Adobe Systems' document \emph{Portable Document Format
    Reference Manual} available at
  \url{http://www.adobe.com/supportservice/devrelations/PDFS/TN/PDFSPEC.PDF}
  on page 27. It can not be reproduced here for strange copyright
  reasons.

  Think of PDF as PostScript without programming constructs. A PDF
  file consists of graphical objects tight together in such a way that
  fast viewing is possible and incremental updates become possible.  }


\question{How_can_I_view_a_PDF_file}{ How can I view a PDF file?}{1998/09/10}

{ \contributor{Steve Phipps}{mailto:slpp@ix.netcom.com}{1998/09/14}
  \contributor{Michael Sanders}{mailto:sanders@umich.edu}{1998/09/17}

  There are several .pdf readers available as freeware over the
  internet.

  Adobe's Acrobat Reader is available for many operating systems,
  including Windows 95, NT, and 3.1, Macintosh, Linux, Sun, and OS/2.
  Download the self-installing executabale from Adobe's website:
  \url{http://www.adobe.com/prodindex/acrobat/}.

  Ghostscipt is a .ps interpreter and Ghostview is its graphical
  front-end.  Ghostscript is also available for many operating
  systems, including UNIX and VMS, MS-DOS, MS-Windows, OS/2 and
  Macintosh.  See the Ghostscript homepage at
  \url{http://www.cs.wisc.edu/~ghost/} for details, documentation, and
  downloads.

  Another previewer is xpdf, a PDF viewer for X maintained by Derek B.
  Noonburg (\url{mailto:derekn@aimnet.com}) with a home page at
  \url{http://www.aimnet.com/~derekn/xpdf}. Xpdf runs under the X
  Window System on UNIX, VMS, and OS/2 and is designed to be small and
  efficient.  It does not use the Motif or Xt libraries and only uses
  standard X fonts.

  [FIXME: some words about Xpdf and other PDF vievers] }


\subsectionf{Authors}{ Authors }


\question{Who_is_the_author_of_pdfTeX}{Who is the author of pdfTeX?}{1998/09/10}

{
\contributor{Pavel Jan\'{\i}k jr}{mailto:Pavel.Janik@inet.cz}{1998/09/10}

  The primary author of pdfTeX is Han The Thanh
  (\url{mailto:thanh@fi.muni.cz}).  [FIXME: in pdftex.ch there is also
  Petr Sojka and the current head (rector) of Masaryk's University -
  prof. Ji\v{r}\'{\i} Zlatu\v{s}ka.]  }


\sectionf{Information}{ Information }
\subsectionf{Locations}{Locations}

\question{Where_can_I_find_pdfTeX}{Where can I find pdfTeX?}{1998/09/10}

{

\contributor{Pavel Jan\'{\i}k jr}{mailto:Pavel.Janik@inet.cz}{1998/09/10}

[FIXME: \url{ftp://ftp.cstug.cz/pub/tex/local/cstug/{janik,thanh}}
  daily mirror on \url{ftp://ftp.inet.cz/pub/Mirrors/pdfTeX} }


\question{What_is_the_latest_version}{ What is the latest version?  }{1998/09/10}

{

\contributor{Pavel Jan\'{\i}k
jr}{mailto:Pavel.Janik@inet.cz}{1998/09/10}

The latest version of pdfTeX is pdfTeX-0.12n. This is the latest
  version of pdfTeX approved by Han The Thanh (for now). After this
  version the primary author (Han The Thanh) left Czech Republic and
  is away for about three or four months. When he comes back, he will
  continue on the pdfTeX's development. In the meantime pdfTeX's
  maintainer is Pavel Jan\'{\i}k ml (he is also student of Masaryk's
  University). He releases bug-fixes to pdfTeX-0.12n as versions
  pdfTeX-0.12o-?, where `?' is a small number. The latest bug-fix
  release is pdfTeX-0.12o-6.  }


\question{Where_can_I_find_some_docs_about_pdfTeX}{ Where can I find
  some docs about pdfTeX?  }{1998/09/10}

{ \contributor{Jody Klymak}{jklymak@apl.washington.edu}{1998/09/18}

A website for the pdfTeX project is maintained by Sebastian Rahtz
  (\url{mailto:s.rahtz@elsevier.co.uk}), and can
  be found at:\\
  \url{http://www.tug.org/applications/pdftex/}\\
  In it you will find:
  \begin{itemize}
    \item The pdfTeX manual in PDF format:\\
    \url{http://www.tug.org/applications/pdftex/pdftexman.pdf}\\
    and HTML format:\\
    \url{http://www.tug.org/applications/pdftex/pdftexman.html}
    \item The pdfTeX mailing list archives:\\
    \url{http://tug.org/ListsArchives/pdftex/threads.html}
    \item This FAQ in PDF format:\\
    \url{http://www.tug.org/applications/pdftex/pdfTeX-FAQ.pdf}\\
    and DVI format:
    \url{http://www.tug.org/applications/pdftex/pdfTeX-FAQ.dvi}
  \end{itemize}

  [FIXME: The best documentation is in the source files...]  }

\question{Is_there_a_pdfTeX_mailing_list}{ Is there a pdfTeX mailing
  list?}{1998/09/10}

{ \contributor{Jody Klymak}{jklymak@apl.washington.edu}{1998/09/18}

Yes, to subscribe send mail to \url{mailto:majordomo@tug.org}, and
  put the line:
\begin{verbatim}
subscribe pdftex username@hostname
\end{verbatim}
  in the body of the message, where username@hostname is your complete
  email address.  }


\question{Where_can_I_find_an_archive_of_the_pdfTeX_mailing_list} {
  Where can I find an archive of the pdfTeX mailing list?  }{1998/09/10}

{ \contributor{Jody Klymak}{jklymak@apl.washington.edu}{1998/09/18}

An ftp site of the mailing list, arranged chronologicaly can be
  found at: \url{ftp://ftp.tug.org/mail-archives/pdftex/} Daily mirror
  is also at \url{ftp://ftp.inet.cz/pub/Mirrors/pdfTeX-MailArchive/} }

\question{How_do_I_get_new_versions_of_this_FAQ}{ How do I get new
  versions of this FAQ?  }{1998/09/10}

{ \contributor{Jody Klymak}{jklymak@apl.washington.edu}{1998/09/18}

  The TeX version of this document is periodically posted to pdfTeX's
  mailing list.

  It is also uploaded to
  \url{http://www.tug.org/applications/pdftex/pdfTeX-FAQ.pdf}
  with its LaTeX source:
  \url{http://www.tug.org/applications/pdftex/pdfTeX-FAQ.tex} }


\question{How_do_I_contribute_to_this_FAQ}{ How do I contribute to
  this FAQ?  }{1998/09/24}

{
\contributor{Jody Klymak}{mailto:jklymak@apl.washington.edu}{1998/09/24}

Send email to the FAQ maintainer: \maintainer with the word
\texttt{FAQ} in the subject line.  If possible FAQ entries should be
formatted like this example:

\begin{verbatim}

\question{This_is_a_template_faq_question}{This is a template faq question}{1998/09/24}

{

\contributor{Pavel Jan\'{\i}k jr}
{mailto:Pavel.Janik@inet.cz}{1998/09/10}
\contributor{Jody Klymak}
{mailto:jklymak@apl.washington.edu}{1998/09/24}

This is a sample FAQ question.  It can reference other
questions, like Question \ref{q:what_is_TeX}, or sections
(See Section \ref{sec:General}).  This FAQ can be found
at \url{http://www.tug.org/applications/pdftex/pdfTeX-FAQ.pdf},
and is maintained by Jody Klymak
\url{mailto:jklymak@apl.washington.edu}

}

\end{verbatim}

}


\sectionf{Installation}{ Installation }

% some subsections for various OSes here
\subsectionf{General_Installation}{General Installation}

\question{How_do_I_install_the_latest_version_of_pdfTeX}{ How do I
  install the latest version of pdfTeX?  }{1998/09/10}

{ \contributor{Pavel Jan\'{\i}k
    jr}{mailto:Pavel.Janik@inet.cz}{1998/09/10}

[FIXME: see script Install and comment it!, add some URLS of web-*,
  web2c-*, ...]  }


\question{How_do_I_use_pdflatex}{ How do I use pdflatex (how to
  generate pdflatex.fmt)?  }{1998/09/10}

{ }


\question{Why_use_pdflatex}{ Why use pdflatex vs (??? i.e.
  latex2html)?  }{1998/09/10}

{ }


\question{How_do_I_compress_my_PDF_files}{ How do I compress my PDF
  files?  }{1998/09/10}

{

\contributor{Pavel Jan\'{\i}k jr}{mailto:Pavel.Janik@inet.cz}{1998/09/10}

pdfTeX can compress its output, but by default pdfTeX does not
  (depending on your configuration). You can manually specify
  compression from 0 to 9 in the source file by the tag
  \verb+\pdfcompresslevel+:
\begin{verbatim}
        \pdfcompresslevel9
\end{verbatim}
  or in the configuration file (pdftex.cfg):
\begin{verbatim}
         compress_level 9
\end{verbatim}
  0 means no compression and 9 is the most (and the slowest)
  compression.  }


\question{What_can_I_do_with_this_pdftex.cfg_file}{ What can I do with
  this pdftex.cfg file?  }{1998/09/10}

{ }


\question{How_can_I_make_a_document_portable_to_both_latex_and_pdflatex}{How
  can I make a document portable to both latex and
  pdflatex}{1998/09/23}

{
\contributor{Christian Kumpf}{mailto:kumpf@igd.fhg.de}{1998/09/23}

Check for the existence of the variable \verb+\pdfoutput+:

\begin{verbatim}

\newif\ifpdf
\ifx\pdfoutput\undefined
   \pdffalse              % we are not running PDFLaTeX
\else
   \pdfoutput=1           % we are running PDFLaTeX
   \pdftrue
\fi

\end{verbatim}

Then use your new variable \verb+\ifpdf+
\begin{verbatim}
\ifpdf
  \usepackage[pdftex]{graphicx}
  \pdfcompresslevel=9
\else
  \usepackage{graphicx}
\fi
\end{verbatim}
}


\sectionf{Fonts_in_pdfTeX}{ Fonts in pdfTeX }


\subsectionf{Fonts_in_general}{ Fonts in general }

\question{What_kind_of_fonts_can_I_use}{ What kind of fonts can I use?
  }{1998/09/10}

{ }


\subsectionf{Type1_fonts}{ Type1 fonts }

\question{How_do_I_use_Type1_fonts}{ How do I use Type1 fonts?  }{1998/09/10}

{ }


\question{How_do_I_generate_TFM_files_for_Type1_fonts}{ How do I
  generate TFM files for Type1 fonts?  }{1998/09/10}

{ }


\subsectionf{TrueType_fonts}{ TrueType fonts }

\question{How_do_I_use_TrueType_fonts}{ How do I use TrueType fonts?
  }{1998/09/10}

{ }


\question{How_do_I_generate_TFM_files_for_TrueType_fonts}{ How do I
  generate TFM files for TrueType fonts?  }{1998/09/10}

{ }


\subsectionf{pk_fonts}{ pk fonts }

\question{How_do_I_use_pk_fonts}{ How do I use pk fonts?
  }{1998/09/10}

{ }


\question{Why_does_Acrobat_Reader_display_pk_fonts_so_poorly}{ Why does
  Acrobat Reader display pk fonts so poorly?  }{1998/09/10}

{ }



\sectionf{Graphics}{ Graphics }


\subsectionf{Graphics_in_general}{ Graphics in general }

\question{How_do_I_include_pictures_in_pdfLaTeX}{ How do I include
  pictures in pdfLaTeX?  }{1998/09/10}

{ \contributor{Jody Klymak}{jklymak@apl.washington.edu}{1998/09/18}

  Pictures come in two formats, vector or bitmapped.  When possible,
  use vector formats when making a PDF document since they can support
  an arbitrary ammount of magnification.  So far, pdfLaTeX supports
  graphics inclusions in PDF, JPEG, PNG, and MetaPost formats [FIXME:
  what others?  Also which are vector or not?]

  As with LaTeX, the best package for image inclusion is
  graphics/graphicx, available on CTAN.  In order to get the graphicx
  package working with pdflatex you must get
  \url{http://tug.org/applications/pdftex/pdftex.def} and put it in
  your TeX tree (mine is at
  \path+C:\TeX\share\texmf\tex\latex\graphics+).  Then some slight
  modifications to your source file:
\begin{verbatim}
%% In the preamble add
\usepackage[pdftex]{graphicx}

%% OPTIONAL: In the main document, immediately after \begin{document}
\DeclareGraphicsExtensions{.jpg,.pdf,.mps,.png}

%% To include a graphics file
\includegraphics{filename_to_include}
\end{verbatim}
  The package will then search for the filename with the above
  extensions if one isn't provided. With this package the image can
  also be scaled or cropped.  This uses D. Carlisle's ``graphics''
  package and is available on CTAN at
  \path+/macros/latex/packages/graphics+.  Note that by not including
  file-extensions in the \verb+\includegraphics+ command, you can
  maintain PDF and DVI versions of your document, especially if you
  include postscript figures (see question
  \ref{q:How_do_I_include_EPS_pictures} and question
  \ref{q:How_can_I_make_a_document_portable_to_both_latex_and_pdflatex})}


\subsectionf{Vector_Formats}{ Vector Formats }

\question{How_do_I_include_PDF_pictures}{ How do I include PDF
  pictures?  }{1998/09/23}

{ \contributor{Carl Zmola}{zmola@campbellsci.com}{1998/09/23}

  In order to include PDF pictures you need pdfLaTeX 0.12n or later.  Make sure that the PDF figure is
  properly cropped. See Question
  \ref{q:How_do_I_convert_an_EPS_figure_to_PDF} for how to do this for
  EPS files.  [FIXME: How do you get Distiller to do this?] Then
  follow the steps outlined in Question
  \ref{q:How_do_I_include_pictures_in_pdfLaTeX}.

}

\question{How_do_I_include_EPS_pictures}{ How do I include EPS
  pictures?  }{1998/09/10}

{ \contributor{Jody Klymak}{jklymak@apl.washington.edu}{1998/09/18}
  You cannot directly do so.  You must convert encapsulated postscript
  pictures to PDF first, explained in question
  \ref{q:How_do_I_convert_an_EPS_figure_to_PDF}, and then include them
  (see question \ref{q:How_do_I_include_PDF_pictures})}

\question{How_do_I_convert_an_EPS_figure_to_PDF}{ How do I convert an
  EPS figure to PDF?  }

{ \contributor {Colin Marquardt} {mailto:colin.marquardt@gmx.de}{1998/09/10}

  You can use epstopdf (a Perl script which uses Ghostscript for
  conversion) at \url{http://tug.org/applications/pdftex/epstopdf} or
  Distiller to do the work for you. [FIXME: add an example here]

  To use epstopdf, you need Perl 5. Usage of epstopdf is easy:
\begin{verbatim}
         epstopdf.pl myfile.eps
\end{verbatim}
  converts your eps-graphic file myfile.eps to the file myfile.pdf.

  Valid options for epstopdf (v2.1) are: [FIXME]

  Change the line
\begin{verbatim}
         $GS="gs";
\end{verbatim}


  in this script to the name of your Ghostscript executable if it is
  different, e.g.
\begin{verbatim}
         $GS="gswin32c";
\end{verbatim}
  on a Win32 system.

  On some systems it is necessary to invoke Perl explicitly, e.g.
  with
\begin{verbatim}
perl epstopdf.pl myfile.eps
\end{verbatim}

[FIXME: How do you use distiller??]
  }


\question{Why_doesnt_my_pdf_picture_show_up_when_I_include_it?}
  { Why doesn't my pdf picture show up when I include it?  }{1998/09/25}

{ \contributor{Carl Zmola}{zmola@acm.org}{1998/09/25}

  You are using Distiler to convert your .eps file to pdf.  Distiller
  does not always set the bounding box correctly. The bounding box of
  an enbedable pdf document must be the page size, and if any part of
  your figure extends beyond the bounding box, the figure will not
  show up.  There are two solutions.

  \begin{enumerate}
    \item{ Get texutil.pl from
        \url{http://www.ntg.nl/context/zipped/texutil.zip} and say
        \begin{verbatim}
            texutil --fig --epspage file.eps
        \end{verbatim}
        and the file is corrected for distiller. (This solution is
        from Hans Hagen \url{mailto:pragma@wxs.nl}.)}
    \item{Instead of running distiller, use
        \url{http://www.tug.org/applications/pdftex/epstopdf} (see
        question \ref{q:How_do_I_convert_an_EPS_figure_to_PDF})}
  \end{enumerate}

  Another tool that might be of some help is \textit{aimaker}.
  \textit{Aimaker} ia a perl script that is designed to convert
  generic .eps files into .aieps files that can be read by adobe
  ilustrator.  \textit{Aimaker} calculates the bounding box for the
  eps file. \textit{Aimaker} is available at
  \url{ftp://ftp.aos.princeton.edu/pub/olszewsk/aimaker.shar}
}

\subsectionf{Bitmap_Formats}{ Bitmap Formats }

\question{How_do_I_include_TIFF_pictures}{ How do I include TIFF
  pictures?  }{1998/09/10}

{ }


\subsectionf{Other_Graphics}{ Other Graphics }

\question{How_can_I_use_MetaPost_in_pdfTeX}{ How can I use MetaPost in
  pdfTeX?  }{1998/09/10}

{ }


\sectionf{Miscellaneous}{ Miscellaneous }

\subsectionf{Kpathsea}{Kpathsea} \question{What_is_this_kpathsea?}{
  What is this kpathsea?  }{ }


\subsectionf{Hyperref}{ Hyperref }

\question{What_is_hyperref}{ What is hyperref?  }{1998/09/10}

{ \contributor{Pavel Jan\'{\i}k jr}{mailto:Pavel.Janik@inet.cz}{1998/09/10}

  The hyperref package is a way of adding hyper-references to a LaTeX
  document.  For instance, a reference to a figure in the text can be
  marked up with a hyper-link, allowing the user to jump to the figure
  without scrolling through a lot of text.  [FIXME: but for now see
  the hyperref package documentation hyperref.dvi or manual.pdf] }

The \texttt{hyperref} package is available on CTAN (or a CTAN mirror)
at
\url{ftp://ftp.ctan.org/tex-archive/macros/latex/contrib/supported/hyperref}


\question{How_can_I_use_it}{ How can I use it?  }{1998/09/10}

{ [FIXME: but for now
  see the hyperref package documentation hyperref.dvi or manual.pdf] }


\question{My_pdftex.cfg_says_to_use_A4...}{ My pdftex.cfg says to use
  A4 paper, but the document comes out in Letter format.  }{1998/09/10}

{ \contributor{Colin Marquardt} {mailto:colin.marquardt@gmx.de}{1998/09/10}
  You certainly use hyperref. This package sets the page dimensions
  according to your settings in LaTeX. If you do not specify
  ``a4paper'' in your options to \verb+\documentclass+, it assumes
  Letter paper and overrides the entries in your \verb+pdftex.cfg+
  file.  }

\question{I_get_the_message...}{ I get the message: ``Warning (ext1):
  destination with the same identifier already exists!''  }{1998/09/10}

{\contributor{Colin Marquardt} {mailto:colin.marquardt@gmx.de}{1998/09/10}
  You get this message if you use hyperref and have some page numbers
  more than once, e.g. when re-starting page numbering with each
  chapter or having an appendix.

  Circumvent this with
\begin{verbatim}
         \usepackage[pdftex,plainpages=false]{hyperref}
\end{verbatim}
  or put
\begin{verbatim}
         plainpages=false
\end{verbatim}
  into your hyperref.cfg.  }

\subsectionf{ConTeXt}{ ConTeXt }


\question{What_is_ConTeXt}{ What is ConTeXt?  }{1998/09/10}

{ \contributor{Pavel Jan\'{\i}k jr}{mailto:Pavel.Janik@inet.cz}{1998/09/10}

  ConTeXt is a full featured macro package that has build in support
  for pdfTeX. More information can be found at
  \url{http://www.ntg.nl/context} (manuals, source code, examples).  }


\question{How_can_I_use_ConTeXt}{ How can I use ConTeXt?  }{1998/09/10}

{ }

\printindex

\end{document}












